\documentclass[]{article}
\usepackage{lmodern}
\usepackage{amssymb,amsmath}
\usepackage{ifxetex,ifluatex}
\usepackage{fixltx2e} % provides \textsubscript
\ifnum 0\ifxetex 1\fi\ifluatex 1\fi=0 % if pdftex
  \usepackage[T1]{fontenc}
  \usepackage[utf8]{inputenc}
\else % if luatex or xelatex
  \ifxetex
    \usepackage{mathspec}
  \else
    \usepackage{fontspec}
  \fi
  \defaultfontfeatures{Ligatures=TeX,Scale=MatchLowercase}
\fi
% use upquote if available, for straight quotes in verbatim environments
\IfFileExists{upquote.sty}{\usepackage{upquote}}{}
% use microtype if available
\IfFileExists{microtype.sty}{%
\usepackage{microtype}
\UseMicrotypeSet[protrusion]{basicmath} % disable protrusion for tt fonts
}{}
\usepackage[margin=1in]{geometry}
\usepackage{hyperref}
\hypersetup{unicode=true,
            pdftitle={Einfache Korpusanalysen: Ein Schnelleinstieg},
            pdfauthor={Stefan Hartmann},
            pdfborder={0 0 0},
            breaklinks=true}
\urlstyle{same}  % don't use monospace font for urls
\usepackage{longtable,booktabs}
\usepackage{graphicx,grffile}
\makeatletter
\def\maxwidth{\ifdim\Gin@nat@width>\linewidth\linewidth\else\Gin@nat@width\fi}
\def\maxheight{\ifdim\Gin@nat@height>\textheight\textheight\else\Gin@nat@height\fi}
\makeatother
% Scale images if necessary, so that they will not overflow the page
% margins by default, and it is still possible to overwrite the defaults
% using explicit options in \includegraphics[width, height, ...]{}
\setkeys{Gin}{width=\maxwidth,height=\maxheight,keepaspectratio}
\IfFileExists{parskip.sty}{%
\usepackage{parskip}
}{% else
\setlength{\parindent}{0pt}
\setlength{\parskip}{6pt plus 2pt minus 1pt}
}
\setlength{\emergencystretch}{3em}  % prevent overfull lines
\providecommand{\tightlist}{%
  \setlength{\itemsep}{0pt}\setlength{\parskip}{0pt}}
\setcounter{secnumdepth}{5}
% Redefines (sub)paragraphs to behave more like sections
\ifx\paragraph\undefined\else
\let\oldparagraph\paragraph
\renewcommand{\paragraph}[1]{\oldparagraph{#1}\mbox{}}
\fi
\ifx\subparagraph\undefined\else
\let\oldsubparagraph\subparagraph
\renewcommand{\subparagraph}[1]{\oldsubparagraph{#1}\mbox{}}
\fi

%%% Use protect on footnotes to avoid problems with footnotes in titles
\let\rmarkdownfootnote\footnote%
\def\footnote{\protect\rmarkdownfootnote}

%%% Change title format to be more compact
\usepackage{titling}

% Create subtitle command for use in maketitle
\providecommand{\subtitle}[1]{
  \posttitle{
    \begin{center}\large#1\end{center}
    }
}

\setlength{\droptitle}{-2em}

  \title{Einfache Korpusanalysen: Ein Schnelleinstieg}
    \pretitle{\vspace{\droptitle}\centering\huge}
  \posttitle{\par}
    \author{Stefan Hartmann}
    \preauthor{\centering\large\emph}
  \postauthor{\par}
      \predate{\centering\large\emph}
  \postdate{\par}
    \date{2019-06-10}

\renewcommand{\figurename}{Fig.}
\renewcommand{\contentsname}{Inhalt}

\begin{document}
\maketitle

{
\setcounter{tocdepth}{4}
\tableofcontents
}
\section{Einstieg}\label{einstieg}

Ziel dieses Tutorials ist es, Anfänger*innen einen möglichst
niedrigschwelligen Einstieg in einfache Korpusanalysen zu ermöglichen.
Es ist insbesondere für Studierende gedacht, die z.B. für eine
Seminararbeit eine Korpusrecherche durchführen möchten, aber bislang
noch keine praktische Erfahrung mit korpuslinguistischen Methoden
sammeln konnten. Das Tutorial bietet anhand eines konkreten Beispiels
eine Schritt-für-Schritt-Anleitung, wie man von der Fragestellung zur
Datengewinnung hin zur Analyse der Daten gelangen kann.

Um wirklich einen Schnelleinstieg bieten zu können, muss ich
notwendigerweise vieles vereinfachen. Für Ihre konkrete Korpusstudie
werden Sie daher wahrscheinlich nicht umhinkommen, sich an der einen
oder anderen Stelle tiefer einzulesen. Dafür verweise ich im Text immer
wieder auf weiterführende Ressourcen. Teilweise finden sich auch in
diesem Tutorial vertiefende Passagen, die Sie (in der HTML-Version)
aufklappen können:

 ‣ \textbf{klick mich}

Hallo, ich bin eine vertiefende Passage.

Sonst gibt es hier nichts zu sehen. Sie können mich gern wieder
schließen. Danke.

Ein Hinweis vorab: Das Tutorial setzt keine Kenntnisse in der
Korpuslinguistik oder im Umgang mit Tabellenkalkulationsprogrammen
voraus, wohl aber grammatische Grundkenntnisse. Sollten Sie die
Fachbegriffe nicht verstehen, empfehle ich sehr, sie nachzuschlagen und
die entsprechenden Wissenslücken zu schließen.

\begin{center}\includegraphics[width=0.3\linewidth,height=0.3\textheight]{fig/by-sa} \end{center}

Dieses Tutorial ist lizenziert unter CC-BY-SA und kann gerne mit
Quellenangabe weitergegeben und adaptiert werden.

\section{Von der Fragestellung zur
Konkordanz}\label{von-der-fragestellung-zur-konkordanz}

Die meisten empirischen Studien lassen sich auf folgende Schritte
herunterbrechen:

\begin{itemize}
\tightlist
\item
  Eine Fragestellung formulieren
\item
  Daten erheben
\item
  Daten auswerten.
\end{itemize}

\subsection{Eine Fragestellung
formulieren}\label{eine-fragestellung-formulieren}

Der erste Schritt ist wahrscheinlich der wichtigste. Nur wenn Sie eine
gute Forschungsfrage haben, können Sie eine aussagekräftige empirische
Analyse durchführen. Aus der Forschungsfrage ergibt sich die Methode:
Für manche Fragestellungen bietet sich z.B. eine Fragebogenstudie an,
für eine eine psycho- oder neurolinguistische Herangehensweise, für
wieder andere eine Korpusrecherche.

Das heißt auch: Wenn Sie eine Korpusanalyse durchführen möchten,
brauchen Sie eine Fragestellung, die korpuslinguistisch
operationalisierbar ist. Beispielsweise lässt sich eine Frage wie
``Welche Gehirnareale werden beim Hören von Bewegungsverben aktiviert?''
natürlich nicht mit Hilfe von Korpusdaten beantworten.

Für unsere Beispielanalyse werfen wir einen Blick auf die prädikative
Verwendung der Partizipien \emph{programmiert} und
\emph{vorprogrammiert}. Letzteres ist manchen Sprachpflegern ein Dorn im
Auge: So bezeichnet es Batian Sick als

\begin{quote}
``umgangssprachliches Blähwort, über das schon Heerscharen von
Sprachpflegern hergefallen sind -- vergebens, denn es wird immer munter
weiter vorprogrammiert. Dabei wissen nicht nur Programmierer: Man
programmiert immer im Voraus, die Vorsilbe vor- ist daher pleonastisch,
zu Deutsch: doppelt gemoppelt.'' \hfill ---
\url{https://bastiansick.de/kolumnen/abc/vorprogrammiertprogrammiert/}
\end{quote}

Was Sprachpfleger wie Sick jedoch oft verkennen, ist, dass Sprache nicht
immer ``logisch'' ist. Vielmehr suchen sich Wörter oft eigene Nischen.
Beispielsweise ist mein Bürostuhl kein \emph{Rollstuhl}, obwohl er
Rollen hat - denn das Wort \emph{Rollstuhl} hat eine eigene Bedeutung
angenommen, die sich nicht kompositional aus seinen Einzelteilen ergibt.
Im Falle von \emph{vorprogrammiert} hingegen passt zwar die Paraphrase
`im Voraus programmiert'. Aber trotzdem wäre denkbar, dass das Wort eine
Spezialisierung erfahren hat: Wird \emph{programmiert} möglicherweise
eher dann verwendet, wenn ein Programmierungsvorgang im wörtlichen Sinn
gemeint ist, und \emph{vorprogrammiert} eher dann, wenn ein z.B. ein
Skandal oder eine Katastrophe ``vorprogrammiert'' sind? Das ist die
Fragestellung, der wir im Folgenden nachgehen möchten.

 ‣ \textbf{Fragestellungen und Hypothesen}

Die Unterscheidung von \textbf{Fragestellung} und \textbf{Hypothese}
bereitet Anfänger*innen oft Schwierigkeiten. Beide hängen eng zusammen.
In unserem Beispiel könnte man die Frage in eine Hypothese
umformulieren: ``vorprogrammiert wird eher in metaphorischem und
programmiert eher im wörtlichen Sinn verwendet.''

Hypothesen ergeben sich in der Regel aus konkreten Fragestellungen.
Beispielsweise könnte in einer soziologischen oder
politikwissenschaftlichen Studie die Fragestellung lauten: Welchen
Einfluss hat das Alter auf das Wahlverhalten in Deutschland? Da man zu
diesem Themengebiet aus der bisherigen Forschung und aus der
Alltagserfahrung das eine oder andere schon weiß, kann man begründete
Annahmen darüber treffen, wie die Antwort auf diese Frage aussieht. So
könnte man davon ausgehen, dass z.B. ältere Menschen eher etablierte und
vielleicht auch eher konservative Parteien wählen und dass außerdem bei
Älteren eine höhere Wahlbeteiligung vorliegt. Diese Annahmen nennt man
Hypothesen. Sie werden auf Grundlage der Daten, die man erhebt,
überprüft.

Nicht immer ist es möglich oder notwendig, konkrete Hypothesen zu
formulieren. Gerade bei Phänomenen, über die noch sehr wenig bekannt
ist, bietet es sich manchmal an, \textbf{explorativ}, also
``erkundend'', zu arbeiten. Auch dann gehe ich mit einer Fragestellung
an meine Daten heran, ohne jedoch im Voraus eine Erwartung zu haben, wie
die Antwort auf meine Frage aussehen wird.

\subsection{Daten erheben}\label{daten-erheben}

\subsubsection{Suchsyntax}\label{suchsyntax}

Für die Datenerhebung verwenden wir das DWDS-Kernkorpus des 20.
Jahrhunderts, das über dwds.de zugänglich ist. Wir suchen auf der
Wortebene mit Hilfe von regulären Ausdrücken nach den Formen
\emph{programmiert} und \emph{vorprogrammiert}. Dafür benutzen wir den
Suchstring
\texttt{@programmiert\ \textbar{}\textbar{}\ @vorprogrammiert}. Das
@-Zeichen bedeutet, dass wir genau diese Strings suchen und keine
anderen Wortformen wie \emph{programmierte}, \emph{programmiertes} etc.
Da uns nur die prädikative Verwendung interessiert, brauchen wir die
flektierten Wortformen nicht. Der horizontale Strich \textbar{} ist der
ODER-Operator; dass man ihn hier doppelt setzen muss, ist eine
Besonderheit der DWDS-Suchsyntax.

‣ \textbf{Alternative Suchabfrage mit regulären Ausdrücken} Alternativ
können wir das gleiche Ergebnis auch durch Verwendug regulärer Ausdrücke
erzielen: \texttt{\$w=/(vor)?programmiert/g}. Ich ermutige alle, die
sich mit Korpuslinguistik beschäftigen wollen, sehr, sich mit regulären
Ausdrücken vertraut zu machen. Allerdings unterstützt die
DWDS-Suchsyntax reguläre Ausdrücke derzeit nur in sehr beschränktem
Maße. (Deutlich besser ist in dieser Hinsicht das alternative
Abfrageportal Dstar, das jedoch für Anfänger*innen nur bedingt geeignet
ist.)

 ‣ \textbf{Zur Suche im DWDS und anderswo} - Die Hilfe zur Suche im DWDS
findet sich hier.

\begin{itemize}
\item
  Einen Einstieg in reguläre Ausdrücke bietet z.B.
  regular-expressions.info.
\item
  In den Begleitmaterialien zu meiner ``Deutschen Sprachgeschichte''
  finden sich ebenfalls einige Tutorials zur Suche in einschlägigen
  Korpora.
\item
  Sehr empfehlenswert und erfreulich ausführlich ist außerdem die
  Korpuslinguistik-Seite von Noah Bubenhofer. 
\end{itemize}

\subsubsection{Export}\label{export}

Die Suche liefert uns 88 Treffer, die nun im Browser in ihrem jeweiligen
Kontext dargestellt werden. Diese Daten wollen wir nun exportieren, und
zwar im ``Key Word in Context'' (KWIC)-Format. Damit ist gemeint, dass
der Suchtreffer zusammen mit seinem unmittelbaren Kontext dargestellt
wird. Erfreulicherweise bietet das DWDS eine sehr gute Exportfunktion,
die es erlaubt, Daten im CSV-Format zu speichern.

\begin{figure}
\includegraphics[width=5.37in]{fig/dwdsdownload} \caption{Export aus dem DWDS}\label{fig:dwdsexp}
\end{figure}

Eine solche Sammlung von Korpusbelegen, wie wir sie jetzt exportiert
haben, nennt man in der Korpuslinguistik \textbf{Konkordanz}. Der
Formatname ``CSV'' steht für ``Comma-Separated Values''. Das heißt, in
der Datei sind die einzelnen Werte durch Kommata voneinander abgetrennt.
In einem Texteditor sieht das Ganze so aus wie in \ref{fig:dwdseditor}.
Wie Sie sehen, enthält die Datei neben den Korpusbelegen selbst auch
Metadaten zu den einzelnen Belegen, z.B. zu Autor*in, Titel etc.

\begin{figure}
\includegraphics[width=4.86in]{fig/conc_in_editor} \caption{Konkordanz im Texteditor}\label{fig:dwdseditor}
\end{figure}

Damit können wir zunächst noch wenig anfangen: Wir wollen die Konkordanz
in ein Tabellenkalkulationsprogramm einlesen.

\subsubsection{Import in ein
Tabellenkalkulationsprogramm}\label{import-in-ein-tabellenkalkulationsprogramm}

Wenn Sie Microsoft Excel auf Ihrem Rechner installiert haben, sind die
Default-Einstellungen höchstwahrscheinlich so gesetzt, dass CSV-Dateien
in Excel geöffnet werden, wenn Sie darauf doppelklicken. Warum das keine
gute Idee ist, zeigt der folgende Screenshot \ref{fig:excel1} (rote
Hervorhebungen von mir nachträglich hinzugefügt).

\begin{figure}
\includegraphics[width=5.04in]{fig/conc_in_excel_bad} \caption{Konkordanz bei direktem Öffnen in Excel}\label{fig:excel1}
\end{figure}

Hier sind einige Sonderzeichen verlorengegangen, weil Excel die
Kodierung der Datei nicht richtig erkannt hat. Es gibt mehrere Wege,
diesem Problem zu begegnen. Ich empfehle hier zwei: Einen für
\protect\hyperlink{import-in-excel}{Excel} und einen für die freie
Alternative \protect\hyperlink{import-in-calc}{Calc}.

\hypertarget{import-in-excel}{\paragraph{Import in
Excel}\label{import-in-excel}}

\begin{enumerate}
\def\labelenumi{\arabic{enumi}.}
\item
  Öffnen Sie die Datei in einem Texteditor. Für Windows empfehle ich
  Notepad++, für Mac die kostenlose (und für unsere Zwecke völlig
  ausreichende) Version von BBEdit, für Linux gibt es z.B. Notepadqq.
\item
  Markieren Sie mit Strg+A bzw. Cmd+A den gesamten Text.
\item
  Öffnen Sie ein leeres Tabellenblatt in Excel. Die nächsten Schritte, 4
  bis 7, sind in \ref{fig:importexcel} visualisiert.
\item
  In den meisten Fällen sollten Sie nun einfach mit Strg+V bzw. Cmd+V
  die Daten einfügen könnn. In manchen Fällen müssen Sie jedoch, wie im
  Screencast \ref{fig:importexcel}, die Option ``Paste Special''
  verwenden (dt. ``Inhalte einfügen'') und angeben, dass Sie den
  Unicode-Text einfügen möchten.
\item
  Mit Klick auf das kleine Klemmbrett-Symbol gelangen Sie zum
  Textimport-Assistenten. Hier müssen Sie Excel sagen, wie der
  eingefügte Text strukturiert ist. Auf der ersten Seite sagen Sie, dass
  es sich um einen Text handelt, bei dem die einzelnen Spalten durch ein
  Trennzeichen getrennt sind (``Delimited'') - diese Option ist in der
  Regel schon angewählt. Außerdem teilen Sie Excel hier mit, dass der
  eingefügte Text UTF-8-formatiert ist.
\item
  Auf der nächste Seite des Textimport-Assistenten geben Sie an, dass
  Kommata als Spaltentrenner benutzt werden. Bei den Textqualifizierern
  müssen Sie nichts ändern, da hier schon Anführungszeichen ausgewählt
  sind: Wie Sie in \ref{fig:dwdseditor} sehen können, werden
  Anführungszeichen in der CSV-Datei genutzt, um zusammengehörigen Text
  zusammenzuhalten (denn wären sie nicht da, würde Excel jedes Komma im
  Text für einen Spaltentrenner handeln.)
\item
  Dieser letzte Schritt erübrigt sich meistens, kann aber nicht schaden:
  Zuletzt können Sie noch alle Spalten als ``Text'' formatieren. (Die
  Datumsspalte können Sie prinzipiell auch als ``Datum'' formatieren,
  falls Sie ausschließlich in Excel weiterarbeiten, aber tendenziell
  rate ich davon ab - gerade bei einer späteren Konversion in andere
  Dateiformate kann dabei alles mögliche schiefgehen\ldots{}) Tipp: Um
  alle Spalten auf einmal als ``Text'' zu formatieren, einfach im
  Fenster ganz nach rechts scrollen und mit gedrückter Shift-Taste auf
  die letzte Spalte klicken, dann sind alle Spalten markiert.
\end{enumerate}

\begin{figure}
\includegraphics[width=6.66in]{fig/import_in_excel} \caption{Import in Excel}\label{fig:importexcel}
\end{figure}

\hypertarget{import-in-calc}{\paragraph{Import in
Calc}\label{import-in-calc}}

Öffnet man die Datei im kostenlosen Tabellenkalkulationsprogramm Calc
von LibreOffice (mit Rechtsklick \textgreater{} Öffen mit), so öffnet
sich zunächst automatisch der Textimportassistent. Hier muss man Calc
mitteilen, welches Format die Datei hat. In unserem Fall ist der Text
UTF-8-kodiert, wir haben Kommas als Spaltentrenner und Anführungszeichen
als Textqualifizieren, wie in \ref{fig:calcimport}.

\begin{figure}
\includegraphics[width=0.5\linewidth,height=0.5\textheight]{fig/calc_import} \caption{Import in Calc}\label{fig:calcimport}
\end{figure}

\section{Von der Konkordanz zur
Analyse}\label{von-der-konkordanz-zur-analyse}

Nun haben wir die Konkordanz erfolgreich in ein
Tabellenkalkulationsprogramm importiert. Hier können wir beliebig viele
weitere Spalten hinzufügen. Das können wir nutzen, um die exportierten
Belege mit \textbf{Annotationen} zu versehen.

\subsection{Annotation}\label{annotation}

Versieht man Daten mit zusätzlichen Informationen, so nennt man diesen
Prozess Annotation. In der Korpuslinguistik stellt die Annotation einen
ganz wesentlichen Schritt dar, der gewissermaßen die Brücke schlägt von
der qualitativ-philologischen Analyse einzelner Belege zur quantitativen
Auswertung.

Wir nutzen im Folgenden die Annotation, um unsere Daten in Kategorien zu
unterteilen, die für unsere Fragestellung sinnvoll sind. Dafür müssen
wir uns zunächst darüber im Klaren sein, was wir von unseren Daten
überhaupt wissen wollen, d.h. wir müssen unsere eingangs genannte
Fragestellung operationalisieren.

Zur Erinnerung: Unsere Fragestellung lautet, ob bei prädikativem
Gebrauch \emph{vorprogrammiert} gegenüber \emph{programmiert} bevorzugt
wird, wenn es sich um einen metaphorischen Kontext handelt.

Konkret bedeutet das, dass wir für jeden Datenpunkt folgende Fragen
beantworten müssen:

\begin{enumerate}
\def\labelenumi{\arabic{enumi}.}
\item
  Handelt es sich um eine prädikative Verwendung? - Schon ein kurzer
  Blick auf die Daten zeigt, dass sich notwendigerweise einige
  \textbf{Fehltreffer} eingeschlichen haben: Häufig finden sich z.B.
  Passivkonstruktionen wie \emph{Es gibt jedoch medizinische Gründe, aus
  denen eine Geburt eingeleitet oder sogar programmiert werden muß}. Uns
  interessieren aber nur Fälle, in denen das Partizip selbst das
  Prädikat bildet, also z.B. \emph{Der Computer ist programmiert} und
  \emph{Die Katastrophe war vorprogrammiert}.
\item
  Handelt es sich um eine metaphorische Verwendung? - Während
  beispielsweise Computer oder Roboter im wörtlichen Sinne programmiert
  werden, bezieht sich der Begriff bei Krisen und Katastrophen darauf,
  dass Voraussetzungen geschaffen wurden, die unausweichlich den
  thematisierten unschönen Ausgang zur Folge haben. Es liegt also ein
  metaphorischer Gebrauch vor, bei der Aspekte der Quelldomäne
  ``Technik'' auf eine abstraktere Zieldomäne übertragen werden.
\end{enumerate}

In den nächsten Abschnitten wollen wir uns beiden Fragen etwas genauer
zuwenden.

\subsubsection{Annotation prädikativ
vs.~nicht-prädikativ}\label{annotation-pradikativ-vs.nicht-pradikativ}

Wenn wir Daten annotieren, besteht eine wesentliche Herausforderung
immer in der \textbf{Operationalisierung} konkreter Fragestellungen. In
vielen Fällen ist es so, dass wir die Frage, die uns interessiert, auf
den ersten Blick glauben für jeden Datenpunkt klar beantworten zu
können. Bei genauerem Hinsehen ergeben sich dann aber doch einige
Zweifelsfälle. So ist es auch hier: Um die Frage operationalisieren zu
können, muss man zunächst einmal die Entscheidung treffen, ob man eine
Struktur wie \emph{Der Computer ist programmiert} als Zustandspassiv mit
\emph{sein} als Hilfsverb (analog zum Vorgangspassiv mit \emph{werden}
als Hilfsverb) oder als Konstruktion aus der Kopula \emph{sein} und dem
Partizip II \emph{programmiert} interpretiert. Wir entscheiden uns hier
für Letzteres. Jedoch zeigt dieses Beispiel: Wie wir Daten
interpretieren, hängt oft genug von unserem theoretischen Zugang ab. Das
ist nicht weiter schlimm, sondern liegt in der Natur der Sache -
Wissenschaft kann nie ganz frei von Theorie und nie ganz frei von
Interpretation sein. Wichtig ist, dass die Entscheidung, die wir
treffen, sich gut begründen lässt und konsequent durchgehalten wird.

Wie setzen wir die Annotation nun in unserer Tabelle um? Auch hier zeige
ich wieder Wege für \protect\hyperlink{umsetzung-in-excel}{Excel} und
\protect\hyperlink{umsetzung-in-libreoffice-calc}{Calc}. Gerade die
unten skizzierte Möglichkeit, Daten als ``Tabelle'' zu formatieren,
finde ich persönlich an Excel sehr hilfreich, weshalb ich Excel i.d.R.
bevorzuge. Allerdings halte ich es auch für wichtig, sich in der
Wissenschaft nicht von proprietärer Software oder proprietären
Datenformaten abhängig zu machen, und nicht jede Uni hat eine
Office-Lizenz - deshalb zeige ich auch den Weg mit der freien
Alternative auf.

\hypertarget{umsetzung-in-excel}{\paragraph{Umsetzung in
Excel}\label{umsetzung-in-excel}}

Zunächst empfiehlt es sich, die Tabelle im Excel-Standardformat .xlsx zu
speichern.

Excel bietet die schöne Möglichkeit, Daten als Tabelle zu formatieren.
Das ist über den Reiter Einfügen \textgreater{} Tabelle möglich, wie in
\ref{fig:excelastable} gezeigt. In der Regel erkennt Excel automatisch
die Dimensionen der Tabelle, sodass Sie nur noch anklicken müssen, dass
die Tabelle Überschriften hat, und dann auf ``OK'' klicken können, und
schon sind alle Zellen schön formatiert, und vor allem kann man über die
kleinen Pfeilsymbole oben die einzelnen Spalten nach bestimmten Werten
filtern, was sich im weiteren Verlauf der Arbeit noch als nützlich
erweisen kann. (Letzteres erreicht man auch über Daten \textgreater{}
Filter, aber mit der Tabellen-Option wird das Ganze optisch noch ein
bisschen hübscher, und vor allem muss man keinen neuen Filter setzen,
wenn man eine neue Spalte hinzufügt.)

Um die Belege im Kontext besser lesen zu können, empfiehlt es sich,
zunächst ein paar Feinjustierungen in der Formatierung vorzunehmen. So
können wir Spalten, die wir derzeit nicht benötigen (z.B. alle Spalten
mit Metadaten), zunächst ausblenden. (Nicht löschen! Im Zweifelsfall nie
Spalten löschen, wer weiß, wozu man sie später noch benötigt\ldots{})
Außerdem kann es hilfreich sein, den Text in der Spalte mit dem linken
Kontext rechtsbündig zu formatieren und die Breite der einzelnen Spalten
so anzupassen, dass man den Beleg und ausreichend viel Kontext lesen
kann und doch alle derzeit wichtigen Spalten gleichzeitig auf dem
Bildschirm zu sehen sind. Wenn Sie die HTML-Version dieses Dokuments
lesen, sehen Sie im weiteren Verlauf von Screencast
\ref{fig:excelastable} (nach der Formatierung der Daten als Tabelle),
wie eine solche Feinjustierung aussehen kann.

\begin{figure}
\includegraphics[width=6.66in]{fig/excelastable01} \caption{Formatierung als Tabelle und Hinzufügen einer Annotationsspalte "praedikativ"}\label{fig:excelastable}
\end{figure}

 ‣ \textbf{Zeilenumbruch innerhalb von Tabellenspalten}

In einigen Fällen, in denen man sehr viel Text im linken und rechten
Kontext hat und in denen man für die Annotation auch auf den weiteren
Kontext angewiesen ist, kann es sinnvoll sein, die Tabelle so zu
formatieren, dass innerhalb der Zelle ein Zeilenumbruch vorgenommen
wird. Standardmäßig ist die Tabelle so formatiert, dass jede Zelle nur
eine Zeile hat, und was über die Zelle hinausgeht, wird nicht angezeigt
(ist aber trotzdem noch in den Daten vorhanden). Wenn man durch Klick
auf den Buchstaben oberhalb der Spalte, die man formatieren möchte,
zunächst die ganze Spalte markiert, kann man unter Rechtsklick
\textgreater{} Zellen formatieren im Tab ``Alignment'' (``Ausrichtung'')
die Option ``Wrap text'' (Zeilenumbruch) aktivieren.

\begin{figure}
\includegraphics[width=6.66in]{fig/excellinewrap} \caption{Zeilenumbruch innerhalb von Zellen einschalten}\label{fig:excelwrap}
\end{figure}

Als nächstes fügen wir eine neue Spalte rechts von der letzten
existierenden Spalte hinzu, der wir die Überschrift ``praedikativ''
geben. (Wir könnten auch problemlos den Umlaut verwenden, aber ich neige
dazu, aus Vorsicht alle Sonderzeichen, die Probleme bereiten könnten,
wegzulassen.) Hier tragen wir nun für jeden Datenpunkt ein, ob es sich
um eine prädikative Verwendung handelt oder nicht. Ich verwende hierfür
gern die Werte ``y'' und ``n'', weil sie schön kurz sind. j/n oder
ja/nein gehen natürlich auch.

Um Zeit zu sparen, kann man auch nur einen der beiden Werte annotieren
und dann die leeren Zellen einfach auffüllen, wie in
\ref{fig:excelbulkchange} gezeigt: Hier sind die ``y''-Werte schon
annotiert, alle anderen Zeilen sind leer. Nun filtert man erst die
``praedikativ''-Spalte so, dass nur noch die leeren Zellen zu sehen
sind, indem man die Zellen mit dem Wert ``y'' abwählt. Dann markiert man
die Spalte ``praedikativ'' von der ersten bis zur letzten Zeile (die
Überschrift wird nicht mitmarkiert). Gibt man nun ``n'' ein (noch nicht
Enter drücken!!), so erscheint der Wert zunächst in der ersten Zeile.
Drückt man nun statt der Eingabetaste Strg+Enter (bzw. bei Mac
Cmd+Enter), so wird der in der ersten Zeile eingegebene Wert auf alle
folgenden Zellen übertragen.

\begin{figure}
\includegraphics[width=6.66in]{fig/excel_bulk_change} \caption{Eine Tabellenspalte wird so gefiltert, dass nur noch die leeren Zellen zu sehen sind, und allen leeren Zellen wird mit Strg/Cmd+Enter derselbe Wert zugewiesen.}\label{fig:excelbulkchange}
\end{figure}

Wenn wir nun den Filter herausnehmen, sehen wir, dass nun alle vorher
leeren Zeilen ein ``n'' haben, während alle Zeilen mit ``y'' unverändert
geblieben sind.

\hypertarget{umsetzung-in-libreoffice-calc}{\paragraph{Umsetzung in
LibreOffice Calc}\label{umsetzung-in-libreoffice-calc}}

In Calc empfiehlt es sich, zunächst einmal die Spaltenbreite anzupassen
und nicht benötigte Spalten auszublenden (nicht löschen - im
Zweifelsfall niemals Spalten oder Zeilen löschen, wer weiß, wofür man
sie noch benötigt!). Ich selbst gehe in der Regel so vor, dass ich alle
Spalten bis auf diejenigen mit den eigentlichen Belegen (linker Kontext,
Treffer, rechter Kontext) ausblende und die Spalte mit dem linken
Kontext so formatiere, dass der Text rechtsbündig angezeigt wird. So
kann ich bequem den Beleg vom linken Kontext über den Treffer bis zum
Keyword lesen. In der HTML-Version dieses Tutorials sehen Sie das in
Screencast \ref{fig:calcformat}.

\begin{figure}
\includegraphics[width=6.66in]{fig/calcformat} \caption{Formatierung der Tabelle in Calc und Setzen eines Filters}\label{fig:calcformat}
\end{figure}

Wenn Sie die Formatierungsoptionen für zukünftige Sitzungen speichern
möchten, müssen Sie die Datei in einem anderen Format, z.B. im
Calc-Standardformat .ods, speichern. Prinzipiell können Sie aber auch
einfach in der CSV-Datei weiterarbeiten. Wenn Sie die Datei
zwischenspeichern, werden dann eventuell neu eingetragene Daten
gespeichert, nicht aber die Formatierung, die Sie dann, wenn Sie die
Datei schließen und wieder öffnen, noch einmal neu einstellen müssen.

Wir können nun eine neue Spalte rechts von der letzten existierenden
Spalte hinzufügen, der wir die Überschrift ``praedikativ'' geben. (Wir
könnten auch problemlos den Umlaut verwenden, aber ich neige dazu, aus
Vorsicht alle Sonderzeichen, die Probleme bereiten könnten,
wegzulassen.) Hier tragen wir nun für jeden Datenpunkt ein, ob es sich
um eine prädikative Verwendung handelt oder nicht. Ich verwende hierfür
gern die Werte ``y'' und ``n'', weil sie schön kurz sind. j/n oder
ja/nein gehen natürlich auch.

Um Zeit zu sparen, kann man auch nur einen der beiden Werte annotieren
und dann die leeren Zellen einfach auffüllen. Dafür müssen wir zunächst
einen Filter setzen, wie in \ref{fig:calcformat} gezeigt. Über diesen
Filter können wir jetzt die leeren Zellen ausblenden, Hier sind die
``y''-Werte schon annotiert, alle anderen Zeilen sind leer. Nun filtert
man erst die ``praedikativ''-Spalte so, dass nur noch die leeren Zellen
zu sehen sind, indem man die Zellen mit dem Wert ``y'' abwählt. Dann
markiert man die Spalte ``praedikativ'' von der ersten bis zur letzten
Zeile (die Überschrift wird nicht mitmarkiert). Gibt man nun ``n'' ein
(noch nicht Enter drücken!!), so erscheint der Wert zunächst in der
ersten Zeile. Drückt man nun statt der Eingabetaste Alt+Enter, so wird
der in der ersten Zeile eingegebene Wert auf alle folgenden Zellen
übertragen.

\begin{figure}
\includegraphics[width=6.66in]{fig/calc_bulkchange} \caption{Eine Tabellenspalte wird so gefiltert, dass nur noch die leeren Zellen zu sehen sind, und allen leeren Zellen wird mit Alt+Enter derselbe Wert zugewiesen.}\label{fig:calcbulkchange}
\end{figure}

Damit ist die Spalte nun vollständig ausgefüllt.

\subsubsection{Annotation metaphorisch
vs.~nicht-metaphorisch}\label{annotation-metaphorisch-vs.nicht-metaphorisch}

Für die weitere Annotation können wir die nicht-prädikativen Fälle außer
Acht lassen. Hier können wir auf die oben erwähnten Filteroptionen
zurückgreifen, um die nicht-prädikativen Fälle herauszufiltern.

Nun gilt es, zu entscheiden, wann \emph{programmiert} und
\emph{vorprogrammiert} metaphorisch verwendet werden und wann nicht.
Auch das ist auf den ersten Blick denkbar einfach: Einen Computer oder
einen Roboter kann man im wörtlichen Sinn programmieren, eine
Katastrophe eher nicht - allenfalls indirekt, indem man z.B. Maschinen
programmiert, die dann die Weltherrschaft übernehmen, siehe so ziemlich
jede Dystopie von ``Terminator'' bis ``Matrix''. Aber genau dieses
indirekte Programmieren bringt uns schon zu möglichen Zweifelsfällen:
Was ist, wenn sich ein Satz wie \emph{Die Konfrontation ist
programmiert} auf einen Roboter bezieht?

Solche Zweifelsfälle ergeben sich gerade bei einer im weitesten Sinne
semantischen Annotation immer. Daher ist es wichtig, klare
\textbf{Annotationsrichtlinien} zu formulieren, in der alle
Annotationsentscheidungen genau dokumentiert sind. Oftmals entwickeln
sich diese Richtlinien im Zuge der Annotation selbst, weil man über
Daten stolpert, die man so zunächst nicht erwartet hätte. (Was übrigens
ein gutes Argument dafür ist, sich bei der Analyse von Sprache nicht
allein auf die eigene Intuition zu verlassen, sondern Korpusdaten zu
Rate zu ziehen!)

Wenn wir nun wörtlichen und metaphorischen Gebrauch annotieren wollen,
könnten unsere Annotationsrichtlinien zunächst ganz einfach so aussehen:

\begin{enumerate}
\def\labelenumi{\arabic{enumi}.}
\item
  Geht aus dem Kontext eindeutig hervor, dass ein Computer bzw. eine
  Maschine programmiert worden ist, liegt wörtlicher Gebrauch vor.
\item
  Geht aus dem Kontext eindeutig hervor, dass sich das Verb auf eine
  andere Entität bezieht, liegt metaphorischer Gebrauch vor.
\item
  Geht aus dem Kontext nicht hervor, worauf genau sich
  ``(vor)programmiert'' ist, wird der Beleg als unklar gewertet.
\end{enumerate}

Auf diesen Kriterien aufbauend können wir nun eine neue Spalte in
unserer Tabelle eröffnen, die wir z.B. ``Lesart'' nennen können. Hier
vergeben wir die Werte ``lit'' (literal/wörtlich), ``met''
(metaphorisch) und ``unklar''. Gerne können Sie es einmal versuchen und
Ihre Ergebnisse dann mit meinen (in den .xlsx- und .ods-Dateien im
``data''-Ordner) vergleichen.

Der große Vorteil der oben formulierten Annotationskriterien ist, dass
sie sich in den meisten Fällen relativ zweifelsfrei anwenden lassen.
Jedoch zeigt sich beim Durchgehen der konkreten Belege, dass die binäre
Unterscheidung ``wörtlich/metaphorisch'' dem Gebrauch von
\emph{(vor)programmiert} möglicherweise nicht ganz gerecht wird. So
fallen die folgenden Beispiele alle in die ``metaphorische'' Kategorie:

\begin{enumerate}
\def\labelenumi{(\arabic{enumi})}
\item
  Der moderne, verbildete Mensch ist nach festen Rhythmen auf das
  eingeschaltete Gerät programmiert und genußbereit.
\item
  Unsere Gene sind auf Lug und Trug programmiert
\item
  da ist Streit mit den Arbeitgebern programmiert.
\end{enumerate}

Die ersten beiden Beispiele bedienen sich der verbreiteten
``Computermetapher'', konzeptualisieren also den menschlichen Geist bzw.
die menschlichen Gene als ``Computer''. Das ist im letzten Beispiel
nicht der Fall: Hier geht es nicht um das Objekt des
Programmiervorgangs, sondern um das Resultat. Diese Verwendung ist in
gewisser Weise also abstrakter. Das ist allerdings eine Dimension, die
grundsätzlich von der Dimension der wörtlichen vs.~metaphorischen
Verwendung unabhängig ist: Angenommen, ich baue mir, wie es verrückte
Wissenschaftler in Filmen gerne tun, eine Frühstücksmaschine, die so
programmiert ist, dass sie mir morgens um 7 ein Spiegelei brät, und
sage: ``Das Spiegelei ist für 7 programmiert'', dann ist das zwar eine
resultatsbezogene, aber keine metaphorische (sondern eher eine
metonymische) Verwendung.

Es wäre daher sinnvoll, auch diese Dimension noch zu kodieren.\footnote{Der
  Vollständigkeit halber sei darauf hingewiesen, dass es sich dabei um
  eine Post-hoc-Analyse handelt. Wenn Sie sich ein wenig in die
  Wissenschaftsphilosophie einlesen, werden Sie merken, dass so etwas
  nicht unumstritten ist: Oft gilt es als Ideal, sämtliche Hypothesen
  und Analysemethoden im Voraus festzulegen, bevor man sich den Daten
  selbst zuwendet. \emph{Post hoc} aufgestellte Hypothesen müsste man
  dann eigentlich anhand von neuen Daten überprüfen. De facto ist es
  freilich oft so, dass für so ein rigides Vorgehen Zeit und Ressourcen
  fehlen. Gerade bei einer Seminararbeit können Sie diesen Punkt
  natürlich in aller Regel getrost ignorieren.} Deshalb fügen wir noch
eine weitere Annotationsspalte hinzu, die wir ``Referenz'' nennen:
Referiert der fragliche Satz auf das, was programmiert wird, oder auf
das Resultat der Programmierung?

Auch hierfür formulieren wir wieder Annotationskriterien:

\begin{enumerate}
\def\labelenumi{\arabic{enumi}.}
\item
  Wenn aus dem Kontext eindeutig hervorgeht, dass sich der Satz auf das
  Objekt des Programmiervorgangs bezieht (\emph{der Computer ist
  programmiert} `jemand (Subj.) hat den Computer (Obj.) programmiert'),
  wird der Beleg mit ``obj'' annotiert.
\item
  Wenn aus dem Kontext eindeutig hervorgeht, dass sich der Satz auf das
  Resultat des (ggf. stark metaphorischen) Programmiervorgangs bezieht
  (\emph{das Spiegelei ist programmiert} `jemand hat die
  Frühstücksmaschine so programmiert, dass sie ein Spiegelei (Resultat)
  macht' oder \emph{die Katastrophe ist programmiert} `es wurden
  Entscheidungen getroffen, die zwangsläufig in eine Katastrophe
  (Resultat) führen'), so wird der Beleg mit ``res'' annotiert.
\item
  Lässt sich keine eindeutige Entscheidung treffen, bekommt der Beleg
  den Wert ``unklar''.
\end{enumerate}

In den .xlsx- und .odt-Dateien im ``data''-Ordner habe ich das in der
Spalte ``Referenz'' umgesetzt. Auch hier können Sie gern die Probe aufs
Exempel machen und überprüfen, ob Ihre Annotationen mit meinen
übereinstimmen. Wahrscheinlich werden Sie im einen oder anderen Fall
andere Entscheidungen treffen als ich - das ist ganz normal und auch der
Grund dafür, warum man idealerweise mindestens zwei Personen unabhängig
voneinander annotieren lassen und dann die Annotationen vergleichen
sollte. (De facto ist das natürlich gerade bei einer Seminararbeit
häufig nicht möglich).

\subsection{Auswertung und
Visualisierung}\label{auswertung-und-visualisierung}

Nachdem wir nun die Daten annotiert haben, können wir unsere
Annotationen quantitativ auswerten. Auch hier zeige ich wieder die
einzelnen Wege für Excel und Calc auf.

\subsubsection{Auswertung und Visualisierung in
Excel}\label{auswertung-und-visualisierung-in-excel}

Ideal für die Auswertung und Visualisierung in Excel ist die
PivotTable-Funktion. Manches an dieser Funktion ist zunächst ein wenig
gewöhnungsbedürftig, aber nach kurzer Eingewöhnungszeit ist sie doch
halbwegs logisch und intuitiv.

Stellen Sie zunächst sicher, dass eine Zelle innerhalb der Tabelle
angewählt ist (z.B. die Zelle ganz oben links). Jetzt klicken wir im
Reiter ``Einfügen'' auf ``PivotTable''. Nun öffnet sich ein Fenster, in
dem wir gefragt werden, welche Zellen Teil der PivotTable werden sollen
(hier sollte Excel bereits automatisch erkannt haben, dass wir die ganze
Tabelle einbeziehen wollen, sodass wir nichts mehr ändern müssen) und ob
die Tabelle auf dem aktuellen oder einem neuen Arbeitsblatt erstellt
werden soll - es empfiehlt sich, sie auf einem neuen Arbeitsblatt zu
erstellen, was auch die Default-Option ist. Also können wir einfach OK
klicken.

Nun öffnet sich ein neues Arbeitsblatt (mit den Reitern unten können Sie
zwischen den Arbeitsblättern navigieren und ihnen ggf. auch
aussagekräftigere Namen geben). Wir sehen ein dreigeteiltes Fenster. Im
Arbeitsblatt selbst finden wir ein etwas kryptisch aussehendes, noch
weitgehend leeres Feld mit einer Beschriftung wie ``PivotTable1'' o.ä.
Das ist quasi der Platzhalter für die noch zu erstellende Tabelle.
Rechts sehen wir oben eine Aufstellung der Namen der Tabellenspalten,
unten sehen wir ein wiederum viergeteiltes Fenster. In die vier Felder
in diesem Fenster können wir nun ausgewählte Spaltennamen aus dem
Fenster oben rechts ziehen. Probieren Sie doch einmal, die Spalte
``Hit'' in das Feld ``Zeilen'' zu ziehen.


\end{document}
